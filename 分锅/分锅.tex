\documentclass[a4paper,12pt]{article}
\usepackage[utf8]{inputenc}
\usepackage{geometry}
\geometry{margin=2.5cm}
\usepackage{titlesec}
\usepackage{longtable}
\usepackage{hyperref}
\usepackage{enumitem}
\usepackage{graphicx}
\usepackage{xcolor}
\usepackage{tcolorbox}
\usepackage{ctex}

\titleformat{\section}{\large\bfseries}{\thesection}{1em}{}

\title{基于 NeRF 与 3D Gaussian Splatting 的物体重建与新视图合成:任务拆解与小组分工}
\author{小组成员:A, B, C}
\date{\today}

\begin{document}

\maketitle

\section{任务要求说明}

\begin{tcolorbox}[colback=gray!5, colframe=black, title=\textbf{课程项目任务要求}]
\textbf{任务 1:基于 NeRF 的物体重建和新视图合成}

基本要求:
\begin{enumerate}
    \item 选取身边的物体拍摄多角度图片/视频,并使用 COLMAP 估计相机参数,随后选用以下其中一种 NeRF 加速技术(如 Plenoxels、TensoRF 等)训练;
    \item 基于训练好的 NeRF 在新的轨迹下渲染环绕物体的视频,并在预留的测试图片上评价定量结果;
    \item 在报告中对选用的 NeRF 变体进行介绍,阐明其与原版 NeRF 的差异。
\end{enumerate}

\vspace{0.5em}
\textbf{任务 2:基于 3D Gaussian Splatting 的物体重建和新视图合成}

基本要求:
\begin{enumerate}
    \item 选取身边的物体拍摄多角度图片/视频,并使用 COLMAP 估计相机参数,随后使用其官方代码库训练 3D Gaussian;
    \item 基于训练好的 3D Gaussian 在新的轨迹下渲染环绕物体的视频,并在预留的测试图片上评价定量结果;
    \item 比较任务 1 中原版 NeRF、任务 1 NeRF 加速技术,以及 3D Gaussian Splatting 三种方法的合成结果和训练/测试效率,在报告中加入相应分析。
\end{enumerate}

\vspace{0.5em}
\textbf{提交要求}:
\begin{enumerate}
    \item 提交 PDF 格式的实验报告,报告中除对模型、数据和实验结果的介绍外,还应包含用 TensorBoard 可视化的训练过程中在训练集和测试集上的 loss 曲线,以及在测试集上的 PSNR 等指标;
    \item 报告中应提供详细的实验设置,如训练测试集划分、网络结构、batch size、learning rate、优化器、iteration、epoch、loss function、评价指标等;
    \item 代码提交到自己的 public GitHub repo,repo 的 README 中应清晰指明如何进行训练和测试,训练好的模型权重和渲染的视频上传到百度云 / Google Drive 等网盘,实验报告内应包含实验代码所在的 GitHub repo 链接及模型权重和视频的下载地址。
\end{enumerate}

\vspace{0.5em}
\textbf{注意事项}:
两个任务都需完成,组队人数少于或等于 3 人(同等质量工作,少于 3 人完成有加分),实验报告由小组成员中一人提交即可。
\end{tcolorbox}

\section{项目概述}
本项目包括两个主要任务:基于 NeRF 的物体重建与新视图合成,以及基于 3D Gaussian Splatting 的方法。每项任务均要求从多视角图像构建三维模型并生成新视角下的视频,同时进行性能评价与对比分析。

\section{任务分解}

\subsection*{总体工作流程}

\begin{itemize}[label=--]
    \item 数据采集:拍摄多角度物体图像或视频
    \item COLMAP重建:估计相机位姿和稀疏点云
    \item 任务1:基于 NeRF 及其加速技术(如 Plenoxels、TensoRF)
    \item 任务2:基于 3D Gaussian Splatting 方法
    \item 新视角渲染与指标评价(如 PSNR)
    \item TensorBoard 训练曲线可视化
    \item 实验报告撰写与代码整理(GitHub + 网盘)
\end{itemize}

\subsection*{子任务拆解表格}

\begin{longtable}{|p{3cm}|p{5.5cm}|p{5.5cm}|}
\hline
\textbf{成员} & \textbf{主要负责内容} & \textbf{交叉协作} \\
\hline
成员 A & 
\begin{itemize}[leftmargin=*]
    \item 数据采集与预处理(视频拍摄、图像转格式)
    \item COLMAP 相机参数重建
    \item NeRF 训练与 TensorBoard 记录
\end{itemize}
&
协助成员 B 进行 NeRF 渲染与轨迹规划 \\
\hline
成员 B &
\begin{itemize}[leftmargin=*]
    \item NeRF 加速技术实现(Plenoxels / TensoRF)
    \item 视频渲染与 PSNR 计算
    \item NeRF 实验部分报告撰写
\end{itemize}
&
将评估结果提供给成员 C 用于对比分析 \\
\hline
成员 C &
\begin{itemize}[leftmargin=*]
    \item 实现 Gaussian Splatting 方法
    \item 对比分析三种方法效果与效率
    \item 撰写实验报告、整理 GitHub 与上传模型权重
\end{itemize}
&
整合 A/B 的训练设置与指标结果 \\
\hline
\end{longtable}

\section{通用子任务}

\begin{enumerate}[label=(\arabic*)]
    \item 拍摄多角度图像或视频,确保物体无遮挡
    \item 使用 COLMAP 生成稀疏点云与相机位姿
    \item 划分训练/测试图片,并统一格式
    \item 统一训练参数(batch size、learning rate、优化器等)
    \item 用 TensorBoard 可视化 loss 曲线
    \item 使用 PSNR、SSIM 或 LPIPS 等指标评估
    \item 将代码上传至 GitHub,并撰写 README
    \item 上传渲染视频与模型到网盘,提供链接
\end{enumerate}

\section{建议与注意事项}

\begin{itemize}
    \item 所有训练流程建议先跑小规模 demo(如 LEGO 数据集)验证流程正确
    \item 建议使用 \LaTeX 或 Overleaf 协作撰写实验报告,格式规范
    \item 每位成员应保留完整实验日志,便于结果整合
    \item 所有训练使用统一轨迹、相同测试集,便于公平比较
\end{itemize}

\end{document}
